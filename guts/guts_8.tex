\documentclass[11pt]{article}
\usepackage[paperwidth=8.5in, paperheight=11in]{geometry}
\usepackage{amsfonts}
\usepackage{amsmath, amsthm, amssymb}
\usepackage{graphicx}
\usepackage{asymptote}
\usepackage{multicol}

%my inputs
\usepackage{framed}
\newtheorem*{definition}{Definition}

\makeatletter
\def\thm@space@setup{%
  \thm@preskip=\parskip \thm@postskip=0pt
}
\makeatother
\theoremstyle{definition}
\newtheorem{problem}{Problem}
\newtheorem*{solution}{Solution}
\newtheorem*{answer}{Answer}
\newtheorem*{solution1}{Solution 1}
\newtheorem*{solution2}{Solution 2}

\def\st{^{\scriptstyle\mbox{st}}}
\def\nd{^{\scriptstyle\mbox{nd}}}
\def\rd{^{\scriptstyle\mbox{rd}}}
\def\th{^{\scriptstyle\mbox{th}}}

\usepackage{fancyhdr}
\pagestyle{fancy}
\cfoot{}
\lhead{\logo}
\rhead{\righthead\vspace{-1em}}
\rfoot{\emph{\sevenpoints}}
\setlength{\headheight}{11em}
\setlength{\headsep}{2em}
\setlength{\voffset}{-1in}
\setlength{\hoffset}{-0.5in}
\addtolength{\textwidth}{1in}
\addtolength{\textheight}{0.25in}
\newlength\FHoffset\setlength\FHoffset{0.5in}
\addtolength\headwidth{2\FHoffset}
\fancyheadoffset{\FHoffset}
\newlength\FHleft\setlength\FHleft{0in}
\newlength\FHright\setlength\FHright{-1in}
\newbox\FHline\setbox\FHline=\hbox{\hsize=\paperwidth%
  \hspace*{\FHleft}%
  \rule{\dimexpr\headwidth-\FHleft-\FHright\relax}{\headrulewidth}\hspace*{\FHright}%
}
\renewcommand\headrule{\vskip-.7\baselineskip\copy\FHline}
\newcommand{\olyinfo}[1]{\begin{flushright} \itshape #1 \end{flushright}\medskip}
\newcommand{\nmbox}[1]{\fbox{\sffamily\LARGE\vphantom y#1} \par\vspace{1em}} % normal box
\newcommand{\fdbox}[2]{\fbox{\sffamily\LARGE\vphantom y#1: \bfseries #2} \par\vspace{1em}} % field box


\begin{document}

\newcommand{\logo}{%
\begin{minipage}[b]{22em}
\centering\noindent
%{\huge\sffamily Intermediate Math Open}
\\[0.5em]
\begin{minipage}[t][4em][t]{12em} \centering
{\huge \bfseries ${\bf 26^{\text{th}}}$ TJIMO } \\
\textsc{\large Alexandria, Virginia}
\end{minipage}
\end{minipage}
\vspace*{-0.05em}
}
\newcommand{\sevenpoints}{}
\newcommand{\righthead}{\fdbox{Round}{Guts}}

\section*{Set 8}
\begin{enumerate}
\setcounter{enumi}{28}
\item Allen starts with the number 0, and wants to get to the number 2014. If on each step, he can either multiply by 3 or add 1, what is the minimum number of steps needed to get to 2014?
\item Let $S_0, S_1, \cdots, S_8$ be subsets of $\{1,2,3,4,5,6,7,8\}$ such that $S_0 = \varnothing$, and $S_i$ is a proper subset of $S_{i+1}$ (i.e. $S_i \neq S_{i+1}$) for integer $i$ with $0 \leq i \leq 7$. How many possible “chains of subsets” (ways to choose the subsets) are there?
\item How many distinct (non-degenerate) kinds of tetrahedrons created from 4 distinct vertices of a cube are there? Two tetrahedrons are not considered distinct if one can be turned into the other by reflection and/or rotation.
\item 
The \emph{rhombicuboctahedron} is a polyhedron with 8 triangular faces and 12 square faces. Each of its 24 vertices has 3 square faces and 1 triangular face meeting at that vertex. Find the volume of a rhombicuboctahedron of side length 1.
\begin{center}
\begin{asy}[viewportwidth=6cm]
import three;
unitsize(1cm);
size(6cm);
size3(6cm,6cm,6cm);
currentprojection=orthographic(5,4,3);

//
// squares
//
draw((1,1,1+sqrt(2))--(1,-1,1+sqrt(2))--(-1,-1,1+sqrt(2))--(-1,1,1+sqrt(2))--cycle);

// draw((1,1,-1-sqrt(2))--(1,-1,-1-sqrt(2))--(-1,-1,-1-sqrt(2))--(-1,1,-1-sqrt(2))--cycle);
draw((1,-1,-1-sqrt(2))--(-1,-1,-1-sqrt(2))--(-1,1,-1-sqrt(2)),dashed);
draw((-1,1,-1-sqrt(2))--(1,1,-1-sqrt(2))--(1,-1,-1-sqrt(2)));

draw((1,1+sqrt(2),1)--(1,1+sqrt(2),-1)--(-1,1+sqrt(2),-1)--(-1,1+sqrt(2),1)--cycle);

// draw((1,-1-sqrt(2),1)--(1,-1-sqrt(2),-1)--(-1,-1-sqrt(2),-1)--(-1,-1-sqrt(2),1)--cycle);
draw((1,-1-sqrt(2),-1)--(-1,-1-sqrt(2),-1)--(-1,-1-sqrt(2),1)--(1,-1-sqrt(2),1),dashed);
draw((1,-1-sqrt(2),-1)--(1,-1-sqrt(2),1));

draw((1+sqrt(2),1,1)--(1+sqrt(2),1,-1)--(1+sqrt(2),-1,-1)--(1+sqrt(2),-1,1)--cycle);
draw((-1-sqrt(2),1,1)--(-1-sqrt(2),1,-1)--(-1-sqrt(2),-1,-1)--(-1-sqrt(2),-1,1)--cycle,dashed);

//
// triangles
//
draw((1,1,1+sqrt(2))--(1,1+sqrt(2),1)--(1+sqrt(2),1,1)--cycle);
draw((-1,1,1+sqrt(2))--(-1,1+sqrt(2),1)--(-1-sqrt(2),1,1)--cycle);
draw((1,-1,1+sqrt(2))--(1,-1-sqrt(2),1)--(1+sqrt(2),-1,1)--cycle);
draw((-1,-1,1+sqrt(2))--(-1,-1-sqrt(2),1)--(-1-sqrt(2),-1,1)--cycle,dashed);
draw((1,1,-1-sqrt(2))--(1,1+sqrt(2),-1)--(1+sqrt(2),1,-1)--cycle);

// draw((-1,1,-1-sqrt(2))--(-1,1+sqrt(2),-1)--(-1-sqrt(2),1,-1)--cycle);
draw((-1,1+sqrt(2),-1)--(-1-sqrt(2),1,-1)--(-1,1,-1-sqrt(2)),dashed);
draw((-1,1+sqrt(2),-1)--(-1,1,-1-sqrt(2)));

// draw((1,-1,-1-sqrt(2))--(1,-1-sqrt(2),-1)--(1+sqrt(2),-1,-1)--cycle);
draw((1,-1,-1-sqrt(2))--(1+sqrt(2),-1,-1)--(1,-1-sqrt(2),-1));
draw((1,-1,-1-sqrt(2))--(1,-1-sqrt(2),-1),dashed);

draw((-1,-1,-1-sqrt(2))--(-1,-1-sqrt(2),-1)--(-1-sqrt(2),-1,-1)--cycle,dashed);

\end{asy}
\end{center}
\end{enumerate}

\eject

\end{document}
