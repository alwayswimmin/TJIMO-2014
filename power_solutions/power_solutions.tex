\documentclass[11pt]{article}
\usepackage[paperwidth=8.5in, paperheight=11in]{geometry}
\usepackage{amsfonts}
\usepackage{amsmath, amsthm, amssymb}
\usepackage{graphicx}
\usepackage{asymptote}
\usepackage{multicol}

%my inputs
\usepackage{framed}
\newtheorem*{definition}{Definition}

\makeatletter
\def\thm@space@setup{%
  \thm@preskip=\parskip \thm@postskip=0pt
}
\makeatother
\theoremstyle{definition}
\newtheorem{problem}{Problem}
\newtheorem*{solution}{Solution}
\newtheorem*{answer}{Answer}
\newtheorem*{solution1}{Solution 1}
\newtheorem*{solution2}{Solution 2}

\def\st{^{\scriptstyle\mbox{st}}}
\def\nd{^{\scriptstyle\mbox{nd}}}
\def\rd{^{\scriptstyle\mbox{rd}}}
\def\th{^{\scriptstyle\mbox{th}}}

\usepackage{fancyhdr}
\pagestyle{fancy}
\cfoot{}
\lhead{\logo}
\rhead{\righthead\vspace{-1em}}
\rfoot{\emph{\sevenpoints}}
\setlength{\headheight}{11em} %previously 16em
\setlength{\headsep}{2em}
\setlength{\voffset}{-1in}
\setlength{\hoffset}{-0.5in}
\addtolength{\textwidth}{1in}
\addtolength{\textheight}{0.25in}
\newlength\FHoffset\setlength\FHoffset{0.5in}
\addtolength\headwidth{2\FHoffset}
\fancyheadoffset{\FHoffset}
\newlength\FHleft\setlength\FHleft{0in}
\newlength\FHright\setlength\FHright{-1in}
\newbox\FHline\setbox\FHline=\hbox{\hsize=\paperwidth%
  \hspace*{\FHleft}%
  \rule{\dimexpr\headwidth-\FHleft-\FHright\relax}{\headrulewidth}\hspace*{\FHright}%
}
\renewcommand\headrule{\vskip-.7\baselineskip\copy\FHline}
\newcommand{\olyinfo}[1]{\begin{flushright} \itshape #1 \end{flushright}\medskip}
\newcommand{\nmbox}[1]{\fbox{\sffamily\LARGE\vphantom y#1} \par\vspace{1em}} % normal box
\newcommand{\fdbox}[2]{\fbox{\sffamily\LARGE\vphantom y#1: \bfseries #2} \par\vspace{1em}} % field box


\begin{document}

\newcommand{\logo}{%
\begin{minipage}[b]{22em}
\centering\noindent
%{\huge\sffamily Intermediate Math Open}
\\[0.5em]
\begin{minipage}[t][4em][t]{12em} \centering
{\huge \bfseries ${\bf 26^{\text{th}}}$ TJIMO } \\
\textsc{\large Alexandria, Virginia}
\end{minipage}
\end{minipage}
\vspace*{-0.05em}
}
\newcommand{\sevenpoints}{}
\newcommand{\righthead}{\fdbox{Round}{Power Solutions}}

\newcounter{problem_count}

\section{Prime Mods}
A \textit{prime} $p$ is defined as a number which has exactly two distinct divisors, 1 and $p$. Prime mods have many unique properties in modular arithmetic.

\begin{enumerate} \addtocounter{enumi}{\value{problem_count}}

\item \addtocounter{problem_count}{1}
\begin{enumerate}
\item Find $x < 5$ such that $2x \equiv 1 \pmod 5$.

\textbf{Solution.}
$x = 3$, as $2 \cdot 3 - 1$ is divisible by 5.

\item Find $x < 5$ such that $3x \equiv 1 \pmod 5$.

\textbf{Solution.}
$x = 2$, as $3 \cdot 2 - 1$ is divisible by 5.

\item Find $x < 5$ such that $4x \equiv 1 \pmod 5$.

\textbf{Solution.}
$x = 4$, as $4 \cdot 4 - 1$ is divisible by 5.

\item For what $a$ can we find an $x$ such that $ax \equiv 1 \pmod{5}$? For what $a$ such that $0 \le a < 5$ does no $x$ exist such that $ax \equiv 1 \pmod{5}$?

\textbf{Solution.}
We can find a corresponding $x$ for $a=2,3,4$ as shown in parts (a), (b), and (c). For $a = 1$, the corresponding $x = 1$ as $1 \cdot 1 - 1$ is divisible by 5. However, there does not exist an $x$ for $a = 0$, as $0 \cdot x = 0$ and $-1$ is not divisible by 5.

\end{enumerate}

\item \addtocounter{problem_count}{1}
Let $a$ be some nonzero number and $p$ some prime. Let the sets $A$, $B$ be defined as
\begin{align*}
A &= \{1, 2, 3,\ldots, p-1\} \\
B &= \{a, 2a, 3a, \ldots, (p-1)a\}.
\end{align*}
\begin{enumerate}
\item Show that no two elements in $B$ are equivalent modulo $p$. (Hint: Recall from the Practice Power that if a prime $p$ evenly divides $ab$, then $p$ must divide at least one of $a$ or $b$.)

\textbf{Solution.}
Suppose two elements in $B$ were equivalent modulo $p$. Then we can write
\[a \cdot i \equiv a \cdot j \pmod{p}\]
for some $i > j$. However, this means that
\[a \cdot (i - j) \equiv 0 \pmod{p}.\]
However, $p$ does not divide either $a$ or $(i - j)$, so this is not possible.

\item How many distinct elements are in $B$ when taken modulo $p$?

\textbf{Solution.}
Since no two elements in $B$ are equivalent modulo $p$, there exist $p - 1$ distinct elements.

\item Show that $A = B$ in modulo $p$. This means $A$ and $B$, in modulo $p$ contain the same elements.

\textbf{Solution.}
Since both $A$ and $B$ are equivalent to exactly the $p - 1$ nonzero residues (that is, numbers) modulo $p$, $A=B$.

\end{enumerate}

\newcounter{set_problem}
\addtocounter{set_problem}{\arabic{problem_count}}

\item \addtocounter{problem_count}{1}
For what $a$ from $\{0, 1, 2, \ldots, p - 1\}$ can we find an $x$ such that $ax \equiv 1 \pmod{p}$ for some prime $p$? Also show that, if we can find such an $x$, the $x$ is unique.

\textbf{Solution.}
For any $a$ from $\{1, 2, \ldots, p - 1 \}$ we can find an $x$ such that $ax \equiv 1 \pmod{p}$, as from problem \arabic{set_problem} the set $\{a, 2a, \ldots, (p - 1) \cdot a\}$ is equivalent to the set $\{1, 2, \ldots p - 1\}$.

Note that this $x$ is unique since all elements in the set $a, 2a, \ldots, (p - 1) \cdot a$ are unique modulo $p$.

For $a = 0$, no such $x$ exists as for any $x$, $ax - 1 = -1$ is not divisible by $p$.

\end{enumerate}

\newcounter{inverse_problem}
\addtocounter{inverse_problem}{\arabic{problem_count}}

Problem \arabic{inverse_problem} has shown that the integers
modulo a prime constitute what is known as a \textit{finite field}. Every nonzero value $a$ in the field has a \textit{multiplicative inverse}, or a number $b$ such that $ab \equiv 1$.

\begin{enumerate}
\addtocounter{enumi}{\value{problem_count}}

\item \addtocounter{problem_count}{1}
\begin{enumerate}
\item Find the smallest positive $n$ such that $2^n \equiv 1 \pmod{3}$.

\textbf{Solution.}
By listing the powers of 2 modulo $3$, we have
\[2 \equiv 2, 4 \equiv 1, \ldots\]
so $n = 2$.

\item 
\begin{enumerate}
\item Find the smallest positive $n$ such that $2^n \equiv 1 \pmod{5}$.

\textbf{Solution.}
By listing the powers of 2 modulo $5$, we have
\[2 \equiv 2, 4 \equiv 4, 8 \equiv 3, 16 \equiv1 \ldots\]
so $n = 4$.

\item Find the smallest positive $n$ such that $3^n \equiv 1 \pmod{5}$.

\textbf{Solution.}
By listing the powers of 3 modulo $5$, we have
\[3 \equiv 3, 9 \equiv 4, 27 \equiv 2, 81 \equiv1 \ldots\]
so $n = 4$.

\item Find the smallest positive $n$ such that $4^n \equiv 1 \pmod{5}$.

\textbf{Solution.}
By listing the powers of 3 modulo $5$, we have
\[3 \equiv 3, 9 \equiv 4, 27 \equiv 2, 81 \equiv1 \ldots\]
so $n = 4$.

\item Find the smallest positive $n$ such that $a^n \equiv 1 \pmod{5}$ for all $a$ not divisible by 5.

\textbf{Solution.}
From previous parts, $n = 4$ satisfies $a=2,3,4$. For $a = 1$, we see $1^4 \equiv 1 \pmod{5}$. Because multiplication is preserved under modular arithmetic, this result can be extended to all $a$ not divisible by 5.

\end{enumerate}
\item
\begin{enumerate}
\item For every integer $a$ in the set $\{1,2,3,4,5,6\}$,
find the smallest positive integer $n$ such that $a^n \equiv 1 \pmod{7}$.

\textbf{Solution.}
By listing powers of numbers from $\{1,2,3,4,5,6\}$, we see $n = 6$ satisfies
\[1^6 \equiv 2^6 \equiv 3^6 \equiv 4^6 \equiv 5^6 \equiv 6^6 \equiv 1 \pmod{7}.\]

\item Find the smallest positive $n$ such that $a^n \equiv 1 \pmod{11}$ for all $a$ not divisible by 11. Compare this with your result from part (b). What do you notice?

\textbf{Solution.}
By listing powers of numbers from $\{1,2,3,4,5,6,7,8,9,10\}$, we see $n = 10$ satisfies
\[1^{10} \equiv 2^{10} \equiv 3^{10} \equiv \ldots \equiv 10^{10} \equiv 1 \pmod{11}.\]

In both cases, $n = p - 1$.

\end{enumerate}
\end{enumerate}

\end{enumerate}

\section{Fermat's Little Theorem}

Fermat's Little Theorem states that $a^{p-1} \equiv 1 \pmod{p}$ for
all primes $p$ and all integers $a$ not divisible by $p$. In
this section you will put together the steps above to prove
Fermat's Little Theorem.

\begin{enumerate} \addtocounter{enumi}{\value{problem_count}}

\item \addtocounter{problem_count}{1}
Use problem \arabic{set_problem} to show that $(p-1)! \equiv a^{p-1}(p-1)! \pmod{p}$.

\textbf{Solution.}
We multiply all the numbers in each set together. However, we know the two sets are equivalent modulo $p$. Therefore the products must be equivalent modulo $p$ as well. Thus
\[(p - 1)! \equiv a^{p - 1}(p - 1)! \pmod{p}.\]

\item \addtocounter{problem_count}{1}
(Fermat.) Show that $a^{p-1} \equiv 1 \pmod{p}$.

\textbf{Solution.}
By rearranging the previous part, we have
\begin{align*}
a^{p - 1}(p - 1)! - (p - 1)! &\equiv 0 \pmod{p} \\
(a^{p - 1} - 1)(p - 1)! &\equiv 0 \pmod{p}.
\end{align*}
Since $p$ does not divide $(p - 1)!$, we have $p$ must divide $a ^{p - 1} - 1$. Therefore
\[a^{p-1} \equiv 1 \pmod{p}.\]

\item \addtocounter{problem_count}{1}
\begin{enumerate}
\item 
Compute the remainder when $4^{45}$ is divided by 43.

\textbf{Solution.}
By Fermat,
\begin{align*}
4^{45} &\equiv 4^{43} \cdot 4^2 \\
&\equiv 1 \cdot 4^2 \\
&\equiv 16  \pmod{43}.
\end{align*}

\item  
Compute the remainder when $5^{1000}$ is divided by 7.

\textbf{Solution.}
By Fermat,
\begin{align*}
5^{1000} &\equiv 5^{996} \cdot 5^4 \\
&\equiv (5^6)^{166} \cdot 5^4 \\
&\equiv 1^{166} \cdot 5^4 \\
&\equiv 625 \equiv 2\pmod{7}.
\end{align*}

\end{enumerate}
\end{enumerate}

\section{Wilson's Theorem}
Wilson's Theorem states that $(n - 1)! \equiv -1 \pmod{n}$ if and only if $n$ is prime. In this section you will prove Wilson's Theorem from the steps above.

\begin{enumerate} \addtocounter{enumi}{\value{problem_count}}

\item \addtocounter{problem_count}{1}
Verify Wilson's Theorem is true for $n = 5$ and $n = 6$.

\textbf{Solution.}
For $n = 5$, $4! \equiv 24 \equiv -1 \pmod{5}$, which confirms Wilson's Theorem as 5 is prime.

For $n = 6$, $5! \equiv 120 \not\equiv -1 \pmod{6}$, which confirms Wilson's Theorem as 6 is composite.

\item \addtocounter{problem_count}{1}
First prove the only if direction:  $(n - 1)! \not\equiv -1 \pmod{n}$ if $n$ is composite.

\textbf{Solution.}
If $n$ is composite, it is either a square of a prime or can be expressed as $a \cdot b$ for $a < b \le n - 1$. In the latter case, it is easy to see that $(n - 1) \equiv 0 \not\equiv 1 \pmod{n}$.

In the former case, Let $n = q^2$ for some prime $q$. $(q^2 - 1)!$ is divisible by $q$ exactly $q - 1$ times. If $q = 2$, then $(4 - 1)! \equiv 2 \pmod{4}$. Otherwise, $(q^2 - 1)!$ is again divisible by $q^2$, and $(n - 1)! \equiv 0 \pmod{n}$.

\item \addtocounter{problem_count}{1}
Now let's try the if direction: $(p - 1)! \equiv -1 \pmod{p}$ for all primes $p$. First we'll split cases where $p$ is odd and $p$ is even. Prove Wilson's Theorem for all even primes $p$.

\textbf{Solution.}
The only even prime is $p = 2$. Then we need only confirm Wilson's Theorem for $p = 2$, which follows since $1 \equiv -1 \pmod{2}$.

\end{enumerate}

Recall from problem \arabic{inverse_problem} we know every integer from $\{1, 2, \ldots, p - 1\}$ has a unique multiplicative inverse modulo prime $p$. That means for a number $a$ in $\{1, 2, \ldots, p - 1\}$, there exists exactly one number $b$ also in $\{1, 2, \ldots, p - 1\}$ such that $ab \equiv 1 \pmod{p}$.

\begin{enumerate} \addtocounter{enumi}{\value{problem_count}}

\item \addtocounter{problem_count}{1}
\begin{enumerate}
\item Find the multiplicative inverses of 1 and $p - 1$ modulo $p$.

\textbf{Solution.}
The multiplicative inverse of $1 \pmod{p}$ is $1$.

The multiplicative inverse of $p - 1 \pmod{p}$ is $p - 1$, since $(p - 1)^2 \equiv p^2 - 2p + 1 \equiv 1 \pmod{p}$.

\item Split the numbers $\{ 2, 3, 4, 5, 6, 7, 8, 9\}$ into four pairs, where each pair of numbers consists of multiplicative inverses modulo 11.

\textbf{Solution.}
The following pairing of the numbers works:
\[(2, 6), (3, 4), (5, 9), (7, 8).\]

\item Show that the numbers from $\{ 2, 3, \ldots, p - 2\}$ can be split into pairs where each pair consists of multiplicative inverses modulo $p$, where $p$ is an odd prime.

\textbf{Solution.}
Every number from the set has a unique multiplicative inverse. Furthermore, this multiplicative inverse is not the number itself, as otherwise we have
\begin{align*}
a^2 &\equiv 1 \pmod{p} \\
a^2 - 1 &\equiv 0 \pmod{p} \\
(a - 1)(a + 1) &\equiv 0 \pmod{p}
\end{align*}
so $p$ must divide either $a - 1$ or $a + 1$, but since $2 \le a \le p - 2$ it cannot be either. Thus we can pair all the numbers together.

\end{enumerate}

\item \addtocounter{problem_count}{1}
Show that $(p - 1)! \equiv -1 \pmod{p}$ for odd primes $p$.

\textbf{Solution.}
Since we can pair all numbers from $\{2, 3, \ldots, p - 2\}$ together, we have
\begin{align*}
(p - 1)! &\equiv 1 \cdot (1 \cdot 1 \cdot \cdots \cdot 1) \cdot (p - 1) \\
&\equiv p - 1 \\
&\equiv -1 \pmod{p}.
\end{align*}

\end{enumerate}

\end{document}

