\documentclass[11pt]{article}
\usepackage[paperwidth=8.5in, paperheight=11in]{geometry}
\usepackage{amsfonts}
\usepackage{amsmath, amsthm, amssymb}
\usepackage{graphicx}
\usepackage{asymptote}
\usepackage{multicol}

%my inputs
\usepackage{framed}
\newtheorem*{definition}{Definition}

\makeatletter
\def\thm@space@setup{%
  \thm@preskip=\parskip \thm@postskip=0pt
}
\makeatother
\theoremstyle{definition}
\newtheorem{problem}{Problem}
\newtheorem*{solution}{Solution}
\newtheorem*{answer}{Answer}
\newtheorem*{solution1}{Solution 1}
\newtheorem*{solution2}{Solution 2}

\def\st{^{\scriptstyle\mbox{st}}}
\def\nd{^{\scriptstyle\mbox{nd}}}
\def\rd{^{\scriptstyle\mbox{rd}}}
\def\th{^{\scriptstyle\mbox{th}}}

\usepackage{fancyhdr}
\pagestyle{fancy}
\cfoot{}
\lhead{\logo}
\rhead{\righthead\vspace{-1em}}
\rfoot{\emph{\sevenpoints}}
\setlength{\headheight}{11em} %previously 16em
\setlength{\headsep}{2em}
\setlength{\voffset}{-1in}
\setlength{\hoffset}{-0.5in}
\addtolength{\textwidth}{1in}
\addtolength{\textheight}{0.25in}
\newlength\FHoffset\setlength\FHoffset{0.5in}
\addtolength\headwidth{2\FHoffset}
\fancyheadoffset{\FHoffset}
\newlength\FHleft\setlength\FHleft{0in}
\newlength\FHright\setlength\FHright{-1in}
\newbox\FHline\setbox\FHline=\hbox{\hsize=\paperwidth%
  \hspace*{\FHleft}%
  \rule{\dimexpr\headwidth-\FHleft-\FHright\relax}{\headrulewidth}\hspace*{\FHright}%
}
\renewcommand\headrule{\vskip-.7\baselineskip\copy\FHline}
\newcommand{\olyinfo}[1]{\begin{flushright} \itshape #1 \end{flushright}\medskip}
\newcommand{\nmbox}[1]{\fbox{\sffamily\LARGE\vphantom y#1} \par\vspace{1em}} % normal box
\newcommand{\fdbox}[2]{\fbox{\sffamily\LARGE\vphantom y#1: \bfseries #2} \par\vspace{1em}} % field box


\begin{document}

\newcommand{\logo}{%
\begin{minipage}[b]{22em}
\centering\noindent
%{\huge\sffamily Intermediate Math Open}
\\[0.5em]
\begin{minipage}[t][4em][t]{12em} \centering
{\huge \bfseries ${\bf 26^{\text{th}}}$ TJIMO } \\
\textsc{\large Alexandria, Virginia}
\end{minipage}
\end{minipage}
\vspace*{-0.05em}
}
\newcommand{\sevenpoints}{Time limit: 45 minutes.}
\newcommand{\righthead}{\fdbox{Round}{Power}}

\newcounter{problem_count}

\section{Prime Mods}
A \textit{prime} $p$ is defined as a number which has exactly two distinct divisors, 1 and $p$. Prime mods have many unique properties in modular arithmetic.

\begin{enumerate} \addtocounter{enumi}{\value{problem_count}}

\item \addtocounter{problem_count}{1}
\begin{enumerate}
\item Find $x < 5$ such that $2x \equiv 1 \pmod 5$.

\item Find $x < 5$ such that $3x \equiv 1 \pmod 5$.

\item Find $x < 5$ such that $4x \equiv 1 \pmod 5$.

\item For what $a$ can we find an $x$ such that $ax \equiv 1 \pmod{5}$? For what $a$ such that $0 \le a < 5$ does no $x$ exist such that $ax \equiv 1 \pmod{5}$?

\end{enumerate}

\item \addtocounter{problem_count}{1}
Let $a$ be some nonzero number and $p$ some prime. Let the sets $A$, $B$ be defined as
\begin{align*}
A &= \{1, 2, 3,\ldots, p-1\} \\
B &= \{a, 2a, 3a, \ldots, (p-1)a\}.
\end{align*}
\begin{enumerate}
\item Show that no two elements in $B$ are equivalent modulo $p$. (Hint: Recall from the Practice Power that if a prime $p$ evenly divides $ab$, then $p$ must divide at least one of $a$ or $b$.)

\item How many distinct elements are in $B$ when taken modulo $p$?

\item Show that $A = B$ in modulo $p$. This means $A$ and $B$, in modulo $p$ contain the same elements.

\end{enumerate}

\newcounter{set_problem}
\addtocounter{set_problem}{\arabic{problem_count}}

\item \addtocounter{problem_count}{1}
For what $a$ from $\{0, 1, 2, \ldots, p - 1\}$ can we find an $x$ such that $ax \equiv 1 \pmod{p}$ for some prime $p$? Also show that, if we can find such an $x$, the $x$ is unique.

\end{enumerate}

\newcounter{inverse_problem}
\addtocounter{inverse_problem}{\arabic{problem_count}}

Problem \arabic{inverse_problem} has shown that the integers
modulo a prime constitute what is known as a \textit{finite field}. Every nonzero value $a$ in the field has a \textit{multiplicative inverse}, or a number $b$ such that $ab \equiv 1$.

\begin{enumerate}
\addtocounter{enumi}{\value{problem_count}}

\item \addtocounter{problem_count}{1}
\begin{enumerate}
\item Find the smallest positive $n$ such that $2^n \equiv 1 \pmod{3}$.

\item 
\begin{enumerate}
\item Find the smallest positive $n$ such that $2^n \equiv 1 \pmod{5}$.

\item Find the smallest positive $n$ such that $3^n \equiv 1 \pmod{5}$.

\item Find the smallest positive $n$ such that $4^n \equiv 1 \pmod{5}$.

\item Find the smallest positive $n$ such that $a^n \equiv 1 \pmod{5}$ for all $a$ not divisible by 5.

\end{enumerate}
\item
\begin{enumerate}
\item For every integer $a$ in the set $\{1,2,3,4,5,6\}$,
find the smallest positive integer $n$ such that $a^n \equiv 1 \pmod{7}$.

\item Find the smallest positive $n$ such that $a^n \equiv 1 \pmod{11}$ for all $a$ not divisible by 11. Compare this with your result from part (b). What do you notice?

\end{enumerate}
\end{enumerate}

\end{enumerate}

\section{Fermat's Little Theorem}

Fermat's Little Theorem states that $a^{p-1} \equiv 1 \pmod{p}$ for
all primes $p$ and all integers $a$ not divisible by $p$. In
this section you will put together the steps above to prove
Fermat's Little Theorem.

\begin{enumerate} \addtocounter{enumi}{\value{problem_count}}

\item \addtocounter{problem_count}{1}
Use problem \arabic{set_problem} to show that $(p-1)! \equiv a^{p-1}(p-1)! \pmod{p}$.

\item \addtocounter{problem_count}{1}
(Fermat.) Show that $a^{p-1} \equiv 1 \pmod{p}$.

\item \addtocounter{problem_count}{1}
\begin{enumerate}
\item 
Compute the remainder when $4^{45}$ is divided by 43.

\item  
Compute the remainder when $5^{1000}$ is divided by 7.


\end{enumerate}
\end{enumerate}

\section{Wilson's Theorem}
Wilson's Theorem states that $(n - 1)! \equiv -1 \pmod{n}$ if and only if $n$ is prime. In this section you will prove Wilson's Theorem from the steps above.

\begin{enumerate} \addtocounter{enumi}{\value{problem_count}}

\item \addtocounter{problem_count}{1}
Verify Wilson's Theorem is true for $n = 5$ and $n = 6$.

\item \addtocounter{problem_count}{1}
First prove the only if direction:  $(n - 1)! \not\equiv -1 \pmod{n}$ if $n$ is composite.

\item \addtocounter{problem_count}{1}
Now let's try the if direction: $(p - 1)! \equiv -1 \pmod{p}$ for all primes $p$. First we'll split cases where $p$ is odd and $p$ is even. Prove Wilson's Theorem for all even primes $p$.

\end{enumerate}

Recall from problem \arabic{inverse_problem} we know every integer from $\{1, 2, \ldots, p - 1\}$ has a unique multiplicative inverse modulo prime $p$. That means for a number $a$ in $\{1, 2, \ldots, p - 1\}$, there exists exactly one number $b$ also in $\{1, 2, \ldots, p - 1\}$ such that $ab \equiv 1 \pmod{p}$.

\begin{enumerate} \addtocounter{enumi}{\value{problem_count}}

\item \addtocounter{problem_count}{1}
\begin{enumerate}
\item Find the multiplicative inverses of 1 and $p - 1$ modulo $p$.

\item Split the numbers $\{ 2, 3, 4, 5, 6, 7, 8, 9\}$ into four pairs, where each pair of numbers consists of multiplicative inverses modulo 11.

\item Show that the numbers from $\{ 2, 3, \ldots, p - 2\}$ can be split into pairs where each pair consists of multiplicative inverses modulo $p$, where $p$ is an odd prime.

\end{enumerate}

\item \addtocounter{problem_count}{1}
Show that $(p - 1)! \equiv -1 \pmod{p}$ for odd primes $p$.

\end{enumerate}

\end{document}

