\documentclass[11pt]{article}
\usepackage[paperwidth=8.5in, paperheight=11in]{geometry}
\usepackage{amsfonts}
\usepackage{amsmath, amsthm, amssymb}
\usepackage{graphicx}
\usepackage{asymptote}
\usepackage{multicol}

%my inputs
\usepackage{framed}
\newtheorem*{definition}{Definition}

\makeatletter
\def\thm@space@setup{%
  \thm@preskip=\parskip \thm@postskip=0pt
}
\makeatother
\theoremstyle{definition}
\newtheorem{problem}{Problem}
\newtheorem*{solution}{Solution}
\newtheorem*{answer}{Answer}
\newtheorem*{solution1}{Solution 1}
\newtheorem*{solution2}{Solution 2}

\def\st{^{\scriptstyle\mbox{st}}}
\def\nd{^{\scriptstyle\mbox{nd}}}
\def\rd{^{\scriptstyle\mbox{rd}}}
\def\th{^{\scriptstyle\mbox{th}}}

\usepackage{fancyhdr}
\pagestyle{fancy}
\cfoot{}
\lhead{\logo}
\rhead{\righthead\vspace{-1em}}
\rfoot{\emph{\sevenpoints}}
\setlength{\headheight}{11em} %previously 16em
\setlength{\headsep}{2em}
\setlength{\voffset}{-1in}
\setlength{\hoffset}{-0.5in}
\addtolength{\textwidth}{1in}
\addtolength{\textheight}{0.25in}
\newlength\FHoffset\setlength\FHoffset{0.5in}
\addtolength\headwidth{2\FHoffset}
\fancyheadoffset{\FHoffset}
\newlength\FHleft\setlength\FHleft{0in}
\newlength\FHright\setlength\FHright{-1in}
\newbox\FHline\setbox\FHline=\hbox{\hsize=\paperwidth%
  \hspace*{\FHleft}%
  \rule{\dimexpr\headwidth-\FHleft-\FHright\relax}{\headrulewidth}\hspace*{\FHright}%
}
\renewcommand\headrule{\vskip-.7\baselineskip\copy\FHline}
\newcommand{\olyinfo}[1]{\begin{flushright} \itshape #1 \end{flushright}\medskip}
\newcommand{\nmbox}[1]{\fbox{\sffamily\LARGE\vphantom y#1} \par\vspace{1em}} % normal box
\newcommand{\fdbox}[2]{\fbox{\sffamily\LARGE\vphantom y#1: \bfseries #2} \par\vspace{1em}} % field box


\begin{document}

\newcommand{\logo}{%
\begin{minipage}[b]{22em}
\centering\noindent
%{\huge\sffamily Intermediate Math Open}
\\[0.5em]
\begin{minipage}[t][4em][t]{12em} \centering
{\huge \bfseries ${\bf 26^{\text{th}}}$ TJIMO } \\
\textsc{\large Alexandria, Virginia}
\end{minipage}
\end{minipage}
\vspace*{-0.05em}
}
\newcommand{\sevenpoints}{}
\newcommand{\righthead}{\fdbox{Round}{Practice Power Solutions}}



\newcounter{problem_count}

\section{Modulo Arithmetic}
Let $a,b,m$ be integers. $a$ and $b$ are said to be \textit{congruent modulo m}
if $m$ evenly divides $a - b$. We write this as
\[a \equiv b \pmod{m}.\]
For example, $1 \equiv 1 \pmod{5}$ since
$1 - 1 = 0$ is divisible by $5$. Similarly, $1 \equiv 6 \pmod{5}$.

\begin{enumerate} \addtocounter{enumi}{\value{problem_count}}
\item \addtocounter{problem_count}{1}
\begin{enumerate}
\item Show $13 \equiv 1 \pmod{6}$.

\textbf{Solution.} 6 divides $13 - 1 = 12$ evenly twice. Then $13 \equiv 1 \pmod{6}$.

\item Show $14 \equiv 2 \pmod{6}$.

\textbf{Solution.} 6 divides $14 - 2 = 12$ evenly twice. Then $14 \equiv 2 \pmod{6}$.

\item Show $13 + 14 \equiv 1 + 2 \equiv 3 \pmod{6}$.

\textbf{Solution.} 6 divides $(14 + 13) - (2 + 1) = 24$ evenly 4 times. Then $14 + 13 \equiv 2 + 1 \pmod{6}$.

\item Show $13 \times 14 \equiv 1 \times 2 \equiv 2 \pmod{6}$.

\textbf{Solution.} 6 divides $(14 \times 13) - (2 \times 1) = 180$ evenly 30 times. Then $14 \times 13 \equiv 2 \times 1 \pmod{6}$.

\end{enumerate}

\item \addtocounter{problem_count}{1}
Let $a,b,c,d,m$ be integers such that $a \equiv b \pmod{m}$ and $c \equiv d \pmod{m}$.
\begin{enumerate}
\item Show $a + c \equiv b + d \pmod{m}$.

\textbf{Solution.}
Since $a \equiv b \pmod{m}$, then $a$ can be expressed as $b + s \cdot m$ for some integer $s$. Similarly, $c$ can be expressed as $d + t \cdot m$ for some integer $t$.

Then $a + c = b + d + m \cdot (s + t)$, so $m$ evenly divides $(a + c) - (b + d)$ and $a + c \equiv b + d \pmod{m}$.

\item Show $a - c \equiv b - d \pmod{m}$.

\textbf{Solution.}
Since $a \equiv b \pmod{m}$, then $a$ can be expressed as $b + s \cdot m$ for some integer $s$. Similarly, $c$ can be expressed as $d + t \cdot m$ for some integer $t$.

Then $a - c = b - d + m \cdot (s - t)$, so $m$ evenly divides $(a - c) - (b - d)$ and $a + c \equiv b + d \pmod{m}$.

\item Show $a \times c \equiv b \times d \pmod{m}$.

\textbf{Solution.}
Since $a \equiv b \pmod{m}$, then $a$ can be expressed as $b + s \cdot m$ for some integer $s$. Similarly, $c$ can be expressed as $d + t \cdot m$ for some integer $t$.

Then $a \times c = b \times d + m \cdot (ds + bt) + m^2 \cdot st$, so $m$ evenly divides $(a \times c) - (b \times d)$ and $a \times c \equiv b \times d \pmod{m}$.

\end{enumerate}
\end{enumerate}

So far we have seen that addition, subtraction, and multiplication
are preserved under modulo arithmetic. \textit{The same is not
necessarily true for division}.

\begin{enumerate} \addtocounter{enumi}{\value{problem_count}}

\item \addtocounter{problem_count}{1}
We know that if a prime $p$ evenly divides $ab$, then $p$ must divide at least one of $a$ or $b$. (Convince yourself of this!) Use this fact to prove that if
\[mx \equiv nx \pmod{p}\]
for $x$ not divisible by p, then
\[m \equiv n \pmod{p}.\]

\textbf{Solution.}
Since $mx \equiv nx \pmod{p}$, $p$ divides $mx - nx = x \cdot (m - n)$. Since $p$ does not divide $x$, $p$ must divide $m - n$, so $m \equiv n \pmod{p}$.

\item \addtocounter{problem_count}{1}
What if the modulo is not prime?
\begin{enumerate}
\item We know that $10 \equiv 4 \pmod{6}$. Can we divide both sides by 2? Is
$5 \equiv 2 \pmod{6}$?

\textbf{Solution.}
No. $5 \not\equiv 2 \pmod{6}$ since $6$ does not divide $5 - 2 = 3$.

\item We know that $25 \equiv 55 \pmod{6}$. Can we divide both sides by 5? Is
$5 \equiv 11 \pmod{6}$?

\textbf{Solution.}
Yes. $5 \equiv 11 \pmod{6}$ since $6$ does divide $5 - 11 = -6$.

\end{enumerate}
What makes (a) any different from (b)? It turns out we can divide
when the greatest common divisor of the number to be divided and the mod $m$ is 1. For instance,
\[ 5a \equiv 5b \pmod{6} \]
implies
\[ a \equiv b \pmod{6} \]
since $\gcd( a, b ) = 1.$

\item \addtocounter{problem_count}{1}
\begin{enumerate}
\item Find positive $x < 10$ such that $3x \equiv 1 \pmod{10}$.

\textbf{Solution.}
By multiplying the first 9 natural numbers by 3, we have the sequence
\[3, 6, 9, 12, 15, 18, 21, 24, 27.\]
The only $x$ satisfying $3x \equiv 1 \pmod{10}$ is $x = 7$.

\item Find positive $x < 10$ such that $7x \equiv 1 \pmod{10}$.

\textbf{Solution.}
By multiplying the first 9 natural numbers by 7, we have the sequence
\[7, 14, 21, 28, 35, 42, 49, 56, 63.\]
The only $x$ satisfying $7x \equiv 1 \pmod{10}$ is $x = 3$.

\item Find positive $x < 10$ such that $9x \equiv 1 \pmod{10}$.

\textbf{Solution.}
By multiplying the first 9 natural numbers by 9, we have the sequence
\[9, 18, 27, 36, 45, 54, 63, 72, 81.\]
The only $x$ satisfying $9x \equiv 1 \pmod{10}$ is $x = 9$.

\item Can we find $x$ such that $2x \equiv 1 \pmod{10}$?

\textbf{Solution.}
No. If there exists such an $x$, then $2x - 1$ is divisible by 10 and thus divisible by 2. However, $2x$ is even so $2x - 1$ is odd, so 2 cannot divide $2x - 1$.

\item Can we find $x$ such that $5x \equiv 1 \pmod{10}$?

\textbf{Solution.}
No. If there exists such an $x$, then $5x - 1$ is divisible by 10 and thus divisible by 5. However, $5x$ is divisible by 5 and $-1$ is not so $5x - 1$ is never divisibly by 5.

\item For what values of $a$ can we find $x$ such that $ax \equiv 1 \pmod{10}$?

\textbf{Solution.}
Any value $a$ in $\{1, 3, 7, 9\}$ works. Parts (a), (b), and (c) showed 3, 7, 9, while $a=1$ has a solution of $x=1$.

Values of $a$ divisible by 2 or 5 does not work, as $ax - 1$ is never divisible by 2 when $a$ is divisible by 2 and is never divisible by 5 when $a$ is divisble by 5.

Because multiplication is preserved in modular arithmetic, any $a$ that can be expressed as one of $\{1+10n, 3+10n, 7+10n, 9+10n\}$ for some integer $n$ is possible.

\end{enumerate}
\end{enumerate}

\end{document}
